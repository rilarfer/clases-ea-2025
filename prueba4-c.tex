\documentclass[11pt,paper=a4,answers,addpoints]{exam}
\usepackage{graphicx,lastpage,comment}
\usepackage{upgreek}
\usepackage{censor}
\censorruledepth=-.2ex
\censorruleheight=.1ex
\hyphenpenalty 10000
\usepackage[paperheight=10.5in,paperwidth=8.27in,bindingoffset=0in,left=0.8in,right=0.8in, top=1.3in,bottom=1in,headsep=.5\baselineskip]{geometry}
\flushbottom
\usepackage[normalem]{ulem}
\usepackage[utf8]{inputenc}
\usepackage[spanish]{babel}
\renewcommand\ULthickness{2pt}
\setlength\ULdepth{1.5ex}
\renewcommand{\baselinestretch}{1}
\pagestyle{empty}

\pagestyle{headandfoot}
\headrule
\newcommand{\continuedmessage}{%
  \ifcontinuation{\footnotesize Pregunta \ContinuedQuestion\ continua\ldots}{}%
}
\runningheader{\footnotesize Programa de Ingeniería de Sistemas}
{\footnotesize Prueba en Parejas — Versión A}
{\footnotesize Página \thepage\ de \numpages}
\footrule
\footer{\footnotesize }
{}
{\ifincomplete{\footnotesize Pregunta \IncompleteQuestion\ continua en la siguiente página \ldots}
  {\iflastpage{\footnotesize Final del examen}{\footnotesize Por favor vea la siguiente página\ldots}}}
\usepackage{amsfonts,amsmath}
\usepackage{cleveref}
\crefname{figure}{figura}{figuras}
\crefname{question}{pregunta}{preguntas}
\renewcommand\thequestion{\arabic{question}}
\renewcommand{\questionlabel}{\thequestion)}
\renewcommand{\questionshook}{%
  \setlength{\leftmargin}{0pt}%
  \setlength{\labelwidth}{-\labelsep}%
}
\decimalpoint
\nopointsinmargin
\pointpoints{punto}{puntos}
\marginpointname{\points}
\pointformat{\boldmath\themarginpoints}
\bracketedpoints
\usepackage{quoting,xparse}
\pointpoints{punto}{puntos}
\bonuspointpoints{punto extra}{puntos extra}
\totalformat{Pregunta \thequestion: \totalpoints puntos}
\hqword{Pregunta}
\hpgword{Página}
\hpword{Puntos}
\hsword{Puntos obtenidos}
\htword{Total}
\usepackage{circuitikz}
\usepackage{color}
\usepackage{pgfplots,graphicx}
\pgfplotsset{compat=1.18}
\usepackage{mathrsfs}
\newcommand{\Laplace}[1]{\ensuremath{\mathscr{L}{\left\lbrace #1\right\rbrace}}}
\newcommand{\InvLap}[1]{\ensuremath{\mathscr{L}^{-1}{\left\lbrace #1\right\rbrace}}}

% === Fondo activado ===
\usepackage{background}
\backgroundsetup{
  scale=1,
  color=black,
  opacity=1,
  angle=0,
  pages=all,
  position=current page.south west,
  nodeanchor=south west,
  vshift=0mm,
  hshift=0mm,
  contents={\includegraphics[width=\paperwidth,height=\paperheight]{fondo.pdf}}
}

\begin{document}
\noprintanswers
\shorthandoff{<>}
\thispagestyle{empty}

%==================== CABECERA ====================
\begin{center}
    \textit{\textbf{Investigación de Operaciones I — Prueba en Parejas (Versión C)}}\\
    \small Problemas de Transporte: EN, Costo Mínimo y VAM
\end{center}
\vspace{-.1cm}
\noindent
\begin{minipage}[t]{.7\textwidth}%
  {\bfseries Integrante 1}: \makebox[.75\textwidth]{\hrulefill} \par
  {\bfseries Integrante 2}: \makebox[.75\textwidth]{\hrulefill} \par
  {\bfseries Docente}: Ricardo Largaespada
\end{minipage}%
\hfill
\begin{minipage}[t]{.4\textwidth}%
  {\bfseries Curso}: Investigación de Operaciones I\par
  {\bfseries Fecha}: \makebox[2.5cm]{\hrulefill} \par
  {\bfseries Grupo}: \rule{2.8cm}{.4pt}
\end{minipage}
\par\noindent\rule{\textwidth}{1pt}

\noindent\textbf{Instrucciones.} Trabaje en parejas y entregue una sola hoja con \textbf{nombres y sección}. Para cada método, muestre la \textbf{asignación}, el \textbf{costo total} y una breve \textbf{justificación}. No optimice con multiplicadores; sólo la SBF. \hfill Total: 10 puntos.

%==================== INICIO PREGUNTA ====================
\begin{questions}
\question[10] \textbf{Problema de transporte balanceado (Versión C).}

Considere la siguiente matriz de costos \(c_{ij}\), con \textbf{ofertas} y \textbf{demandas}:

\begin{center}
\renewcommand{\arraystretch}{1.15}
\begin{tabular}{c|cccc|c}
 & \multicolumn{4}{c|}{\textbf{Destino}} & \textbf{Oferta}\\
\cline{2-5}
\textbf{Origen} & 1 & 2 & 3 & 4 & \\
\hline
1 & 7 & 8 & 4 & 8 & 6\\
2 & 5 & 5 & 9 & 3 & 2\\
3 & 2 & 5 & 2 & 4 & 13\\
\hline
\textbf{Demanda} & 4 & 6 & 5 & 6 &
\end{tabular}
\end{center}

\begin{parts}
  \part[3] Obtenga la \textbf{SBF por Esquina Noroeste}, indicando celdas básicas, unidades asignadas y \textbf{costo total}.
  \vspace{1.8cm}
  \part[3] Obtenga la \textbf{SBF por Costo Mínimo}, indicando celdas básicas, unidades asignadas y \textbf{costo total}.
  \vspace{1.8cm}
  \part[3] Aplique la \textbf{Aproximación de Vogel (VAM)} y halle la SBF y su \textbf{costo total}.
  \vspace{1.4cm}
  \part[1] Compare las tres soluciones: ¿son \textbf{distintas} las SBF? Justifique brevemente.
  \vspace{0.8cm}
\end{parts}

\end{questions}
\end{document}
