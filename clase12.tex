\documentclass{beamer}

\usepackage[utf8]{inputenc}
\usepackage[spanish]{babel}
\usepackage{amsmath}
\usepackage[nosetup]{evan}
\usetheme{Goddard}
\hypersetup{colorlinks,allcolors=.,urlcolor=magenta}
\usepackage[table]{xcolor} % Para definir colores en tablas
\usepackage{graphicx} % Para redimensionar la tabla

\title{Clase Práctica}
\subtitle{Unidad 2: Introducción a la Teoría de Probabilidades}
\author{Ricardo Jesús Largaespada Fernández}
\institute{Ingeniería de Sistemas, DACTIC, UNI}
\date{\today}

\begin{document}
\frame{\titlepage}

% ===================== 1
\section{Conteo y permutaciones}

\begin{frame}{Tres parejas en una fila}
Seis estudiantes (tres parejas) se sientan en una fila de seis butacas. \\
\textbf{Calcule el número de formas de sentarse} en cada caso:
\begin{enumerate}
  \item Sin restricciones.
  \item Cada pareja debe sentarse junta (por parejas).
  \item Todas las mujeres ocupan las tres butacas de la izquierda y todos los hombres las tres de la derecha.
\end{enumerate}
\medskip\pause
\textbf{Respuestas}
\begin{enumerate}
  \item $6!=720$.
  \item Trato cada pareja como bloque: $3!\,(2!)^3=48$.
  \item Permutar mujeres y hombres por separado: $3!\cdot 3!=36$.
\end{enumerate}
\end{frame}

% ===================== 2
\section{Órdenes aleatorios}

\begin{frame}{Difusión de un rumor}
A quiere contar un rumor a cinco amigos $B,C,D,E,F$. Empieza eligiendo al azar a uno; a partir de ahí, en cada paso alguien que ya lo oyó se lo cuenta a \emph{una persona nueva}, hasta que todos lo oyen. \\
\textbf{Calcule las probabilidades}:
\begin{enumerate}
  \item Que el orden exacto sea $B,C,D,E,F$.
  \item Que $F$ sea la tercera persona en escucharlo.
  \item Que $F$ sea la última en escucharlo.
\end{enumerate}
\medskip\pause
\textbf{Respuestas}
\begin{enumerate}
  \item Orden uniforme de 5 personas: $1/5!=1/120$.
  \item Cualquier persona es igualmente probable en cada posición: $1/5=0.2$.
  \item Idem: $1/5=0.2$.
\end{enumerate}
\end{frame}

% ===================== 3
\section{Muestreo sin reemplazo (Hipergeométrica)}

\begin{frame}{Computadoras defectuosas}
Un lote tiene 50 computadoras, de las cuales 5 son defectuosas. Se extraen 2 \emph{sin reemplazo}. \\
\textbf{Calcule}:
\begin{enumerate}
  \item $P(\text{segunda defectuosa}\mid \text{primera defectuosa})$.
  \item $P(\text{ambas defectuosas})$.
  \item $P(\text{ambas aceptables})$.
\end{enumerate}
\medskip\pause
\textbf{Respuestas}
\begin{enumerate}
  \item Quedan $4$ defectuosas de $49$: $4/49$.
  \item $(5/50)\cdot(4/49)=2/245\approx 0.00816$.
  \item $(45/50)\cdot(44/49)=198/245\approx 0.808$.
\end{enumerate}
\end{frame}

% ===================== 4
\section{Ensayos independientes (Binomial)}

\begin{frame}{Centro de verificación de emisiones}
Cada vehículo pasa con probabilidad $p=0.7$ e independencia entre vehículos. Se inspeccionan 3 vehículos. \\
\textbf{Calcule}:
\begin{enumerate}
  \item Probabilidad de que los tres pasen.
  \item Probabilidad de que al menos uno pase.
  \item Probabilidad de que exactamente uno pase.
  \item Probabilidad de que a lo más uno pase.
  \item Probabilidad de que los tres pasen dado que al menos uno pasa.
\end{enumerate}
\medskip\pause
\textbf{Respuestas}
\begin{enumerate}
  \item $0.7^3=0.343$.
  \item $1-0.3^3=0.973$.
  \item $\binom31(0.7)(0.3)^2=0.189$.
  \item $0.3^3+3(0.7)(0.3)^2=0.216$.
  \item $\dfrac{0.7^3}{1-0.3^3}\approx 0.3525$.
\end{enumerate}
\end{frame}

% ===================== 5
\section{Probabilidad total y condicional}

\begin{frame}{Elección de gasolina y llenar el tanque}
En una gasolinera: $40\%$ usa regular ($A_1$), $35\%$ plus ($A_2$), $25\%$ premium ($A_3$). \\
Probabilidades de llenar el tanque ($B$): $P(B\mid A_1)=0.30$, $P(B\mid A_2)=0.60$, $P(B\mid A_3)=0.50$. \\
\textbf{Calcule}:
\begin{enumerate}
  \item Probabilidad de que el cliente pida plus y llene el tanque.
  \item Probabilidad de que el cliente (sin importar tipo) llene el tanque.
\end{enumerate}
\medskip\pause
\textbf{Respuestas}
\begin{enumerate}
  \item $P(A_2\cap B)=0.35\cdot 0.60=0.21$.
  \item $P(B)=0.40(0.30)+0.35(0.60)+0.25(0.50)=0.455$.
\end{enumerate}
\end{frame}

% ===================== 6
\section{Independencia simple}

\begin{frame}{Permanencia laboral de dos personas}
La probabilidad de permanecer al menos 10 años en la empresa es $1/6$. Dos personas inician el mismo día, de forma independiente. \\
\textbf{Calcule}:
\begin{enumerate}
  \item Probabilidad de que la primera permanezca menos de 10 años.
  \item Probabilidad de que ambas permanezcan menos de 10 años.
  \item Probabilidad de que al menos una permanezca 10 años o más.
\end{enumerate}
\medskip\pause
\textbf{Respuestas}
\begin{enumerate}
  \item $5/6$.
  \item $(5/6)^2=25/36$.
  \item $1-(5/6)^2=11/36$.
\end{enumerate}
\end{frame}

% ===================== 7
\section{Conjuntos e inclusión–exclusión}

\begin{frame}{Resultados en tres ejercicios}
De $N=398$ estudiantes: \\
\quad $|A|=321$ (al menos el 1°), $|B|=210$ (al menos el 2°), $|C|=200$ (al menos el 3°); \\
\quad $|A\cap B|=141$, $|B\cap C|=138$, $|A\cap B\cap C|=83$; \\
\quad $4$ no resolvieron ninguno. \\
\textbf{Calcule las probabilidades}:
\begin{enumerate}
  \item Exactamente el 1° y el 3°.
  \item Exactamente el 1°.
  \item El 1° y el 3° (al menos esos dos).
  \item Al menos dos ejercicios.
  \item A lo sumo dos ejercicios.
\end{enumerate}
\medskip\pause
\textbf{Resultados intermedios}
\[
|A\cup B\cup C|=394 \Rightarrow |A\cap C|=141.
\]
Exactos: $ab=58,\; bc=55,\; ac=58,\; a=122,\; b=14,\; c=4,\; abc=83$. \\
\pause
\textbf{Respuestas} (sobre $398$)
\begin{enumerate}
  \item $58/398=29/199\approx 0.1457$.
  \item $122/398=61/199\approx 0.3065$.
  \item $141/398\approx 0.3543$.
  \item $(58+58+55+83)/398=127/199\approx 0.638$.
  \item $(398-83)/398=315/398\approx 0.791$.
\end{enumerate}
\end{frame}

% ===================== 8
\section{Bayes}

\begin{frame}{Pregunta de opción múltiple}
Una pregunta tiene 4 opciones. La probabilidad de que el estudiante \emph{sepa} la respuesta es $0.8$; si no la sabe, contesta al azar (probabilidad de acierto $0.25$). Si el estudiante \emph{acierta}, \textbf{calcule} la probabilidad de que realmente la supiera.
\medskip\pause

\[
P(\text{sabe}\mid \text{acierta})
=\frac{1\cdot 0.8}{1\cdot 0.8+0.25\cdot 0.2}
=\frac{0.8}{0.85}
=\boxed{\frac{16}{17}\approx 0.9412}.
\]
\end{frame}

% ===================== cierre
\section*{Cierre}
\begin{frame}{Cierre}
\begin{itemize}
  \item Cadena de ideas: permutaciones $\rightarrow$ órdenes aleatorios $\rightarrow$ hipergeométrica/binomial.
  \item Herramientas clave: probabilidad total, independencia, inclusión–exclusión, Bayes.
\end{itemize}
\end{frame}

\end{document}
