\documentclass{beamer}

\usepackage[utf8]{inputenc}
\usepackage[spanish]{babel}
\usepackage{amsmath}
\usepackage[nosetup]{evan}
\usetheme{Goddard}
\hypersetup{colorlinks,allcolors=.,urlcolor=magenta}
\usepackage[table]{xcolor} % Para definir colores en tablas
\usepackage{graphicx} % Para redimensionar la tabla

\title{Estadística Aplicada}
\subtitle{Unidad 1: Estadística Descriptiva}
\author{Ricardo Jesús Largaespada Fernández}
\institute{Ingeniería de Sistemas, DACTIC, UNI}
\date{\today}

\begin{document}
\frame{\titlepage}

\begin{frame}{Objetivo de la clase}
\begin{itemize}
  \item Aplicar las medidas estadísticas de forma analítica y gráfica en la resolución e interpretación de problemas contextualizados.
  \item Interpretar los resultados obtenidos a partir de las medidas de tendencia central, variabilidad y posición, identificando su utilidad en diferentes contextos.
  \item Utilizar representaciones gráficas para apoyar el análisis y la comunicación de conclusiones estadísticas.
\end{itemize}
\end{frame}

\begin{frame}
\frametitle{Agenda}
\tableofcontents
\end{frame}

% ==============================
\section{Medidas de Tendencia Central}
% ==============================
\begin{frame}{Media aritmética}
\textbf{Media aritmética:} valor promedio de los datos.
\[
\bar{x} = \frac{1}{n}\sum_{i=1}^n x_i
\qquad
\mu = \frac{1}{N}\sum_{i=1}^N X_i
\]

\textbf{Datos agrupados:}
\[
\bar{x} = \frac{\sum_{i=1}^{k} f_i X_i}{n}, 
\quad n=\sum f_i
\]

\begin{itemize}
\item $x_i$: datos individuales
\item $X_i$: marca de clase
\item $f_i$: frecuencia absoluta
\end{itemize}
\end{frame}

\begin{frame}{La mediana}
\textbf{Mediana:} valor central que divide la muestra en dos mitades.

\textbf{Datos agrupados:}
\[
M_e = Li + \left(\frac{\tfrac{n}{2}-F}{f}\right)\cdot C
\]

\begin{itemize}
\item $Li$: límite inferior real del intervalo mediano
\item $F$: frecuencia acumulada antes del intervalo
\item $f$: frecuencia del intervalo mediano
\item $C$: amplitud de clase
\end{itemize}

\textbf{Datos no agrupados:}
\[
M_e = 
\begin{cases}
x_{\left(\tfrac{n+1}{2}\right)}, & n \text{ impar}\\[6pt]
\dfrac{x_{\tfrac{n}{2}} + x_{\tfrac{n}{2}+1}}{2}, & n \text{ par}
\end{cases}
\]
\end{frame}

\begin{frame}{La moda}
\textbf{Moda:} valor con mayor frecuencia.

\textbf{Datos agrupados:}
\[
M_o = Li + \left(\frac{\Delta_1}{\Delta_1+\Delta_2}\right)C
\]
\begin{itemize}
\item $\Delta_1 =$ frecuencia modal – frecuencia anterior
\item $\Delta_2 =$ frecuencia modal – frecuencia posterior
\item $C =$ amplitud de clase
\end{itemize}
\end{frame}

\begin{frame}{Formas de la distribución (simetría)}
\centering
\begin{tikzpicture}[scale=1.0]
\draw[->] (-3,0)--(3,0);
\draw plot[smooth,domain=-3:3,samples=100] (\x,{exp(-\x*\x/2)});
\node at (0,-0.3) {$M_o=M_e=\bar{x}$};
\end{tikzpicture}

\begin{itemize}
\item \textbf{Simétrica:} $M_o=M_e=\bar{x}$
\item \textbf{Asimétrica positiva:} $M_o < M_e < \bar{x}$
\item \textbf{Asimétrica negativa:} $\bar{x}<M_e<M_o$
\end{itemize}
\end{frame}

% --- Asimetría positiva (sesgo a la derecha) ---
\begin{frame}{Formas de la distribución: Asimetría positiva}
\centering
\begin{tikzpicture}[x=1cm,y=2.2cm]
  % Eje
  \draw[->] (0,0) -- (8.4,0);

  % Curva tipo Gamma (cola a la derecha)
  \draw[thick] plot[smooth, domain=0:8, samples=240]
    (\x, {(\x/2.2)^2 * exp(-\x/1.2)});

  % Líneas de Mo < Me < x̄
  \draw[densely dashed] (2.0,0) -- (2.0,{(2.0/2.2)^2*exp(-2.0/1.2)});
  \draw[densely dashed] (3.2,0) -- (3.2,{(3.2/2.2)^2*exp(-3.2/1.2)});
  \draw[densely dashed] (4.3,0) -- (4.3,{(4.3/2.2)^2*exp(-4.3/1.2)});

  % Etiquetas
  \node[below] at (2.0,0) {$M_o$};
  \node[below] at (3.2,0) {$M_e$};
  \node[below] at (4.3,0) {$\bar{x}$};
\end{tikzpicture}

\medskip
\small En asimetría positiva: \(M_o < M_e < \bar{x}\).
\end{frame}

% --- Asimetría negativa (sesgo a la izquierda) ---
\begin{frame}{Formas de la distribución: Asimetría negativa}
\centering
\begin{tikzpicture}[x=1cm,y=2.2cm]
  % Eje
  \draw[->] (0,0) -- (8.4,0);

  % Curva espejo (cola a la izquierda)
  \draw[thick] plot[smooth, domain=0:8, samples=240]
    (\x, {((8-\x)/2.2)^2 * exp(-(8-\x)/1.2)});

  % Líneas de x̄ < Me < Mo
  \draw[densely dashed] (2.7,0) -- (2.7,{((8-2.7)/2.2)^2*exp(-(8-2.7)/1.2)});
  \draw[densely dashed] (3.8,0) -- (3.8,{((8-3.8)/2.2)^2*exp(-(8-3.8)/1.2)});
  \draw[densely dashed] (5.0,0) -- (5.0,{((8-5.0)/2.2)^2*exp(-(8-5.0)/1.2)});

  % Etiquetas
  \node[below] at (2.7,0) {$\bar{x}$};
  \node[below] at (3.8,0) {$M_e$};
  \node[below] at (5.0,0) {$M_o$};
\end{tikzpicture}

\medskip
\small En asimetría negativa: \(\bar{x} < M_e < M_o\).
\end{frame}

% ==============================
\section{Medidas de Posición}
% ==============================
\begin{frame}{Cuartiles, deciles y percentiles}
\textbf{Datos agrupados:}
\[
Q_i = Li + \left(\frac{\tfrac{i n}{4}-F}{f}\right)C
\quad (i=1,2,3,4)
\]

\[
D_i = Li + \left(\frac{\tfrac{i n}{10}-F}{f}\right)C
\quad (i=1,2,\dots,10)
\]

\[
P_i = Li + \left(\frac{\tfrac{i n}{100}-F}{f}\right)C
\quad (i=1,2,\dots,100)
\]
\end{frame}

\begin{frame}{Cuartiles en datos no agrupados}
\textbf{Si $n$ es impar:}
\[
Q_2 = x_{\tfrac{n+1}{2}} \quad (\text{mediana})
\]

\textbf{Si $n$ es par:}
\[
Q_2 = \frac{x_{\tfrac{n}{2}} + x_{\tfrac{n}{2}+1}}{2}
\]

\small Los cuartiles $Q_1$ y $Q_3$ se calculan análogamente, considerando las posiciones correspondientes.
\end{frame}

% ==============================
\section{Medidas de Variabilidad, Momentos y Forma}
% ==============================
\begin{frame}{Medidas de variabilidad}
Son medidas que expresan la \textbf{dispersión} de los datos alrededor de un valor promedio. 
Una vez localizado el centro de la distribución, interesa cuantificar qué tanto se
\emph{dispersan} los datos respecto de ese centro.
\medskip

Casi ningún conjunto de datos tiene todos los valores iguales; por eso el grado de
dispersión alrededor del promedio no es cero. Lo medimos principalmente con la
\alert{varianza} y la \alert{desviación estándar}.
\end{frame}

\begin{frame}{Varianza $S^2$ (muestra) y $\sigma^2$ (población)}
\textbf{Datos no agrupados (muestra):}
\[
S^{2}=\frac{\sum_{j=1}^{n}(x_j-\bar{x})^{2}}{n-1}, 
\qquad 
\bar{x}=\frac{1}{n}\sum_{j=1}^{n}x_j.
\]
\textbf{Datos agrupados en clases (muestra):}
\[
S^{2}=\frac{n\sum x_i^{2}f_i-\left(\sum x_i f_i\right)^{2}}{n(n-1)}, 
\quad
n=\sum f_i.
\]
\begin{itemize}
\item $x_j$: dato individual;\; $x_i$: \emph{marca de clase};\; $f_i$: frecuencia.
\item La varianza es siempre un valor \textbf{no negativo}.
\end{itemize}
\end{frame}

\begin{frame}{Desviación estándar y coeficiente de variación}
\textbf{Desviación estándar:} es la raíz cuadrada de la varianza.
\[
S=\sqrt{S^{2}} \quad (\text{muestra}), 
\qquad 
\sigma=\sqrt{\sigma^{2}} \quad (\text{población}).
\]

\textbf{Coeficiente de variación (CV):} compara dispersiones en escalas distintas.
\[
C_V=\frac{S}{\bar{x}}.
\]
Se prefiere la distribución con \(\,C_V\,\) menor (más representativa / menos variabilidad relativa).
\end{frame}

% ==============================
\subsection{Momentos}
% ==============================
\begin{frame}{Momentos (no centrados y centrados)}
Sea \(y_1,\dots,y_n\) una variable cuantitativa.
\medskip

\textbf{Momento (no centrado) de orden \(i\) respecto al origen:}
\[
\hat{y}^{\,i}=\frac{1}{n}\sum_{j=1}^{n} y_j^{\,i}.
\]
\textbf{Momento (centrado) de orden \(i\) respecto a la media \(\bar{y}\):}
\[
M_i=\frac{1}{n}\sum_{j=1}^{n}\bigl(y_j-\bar{y}\bigr)^{i}.
\]
Casos importantes:
\[
M_1=0, 
\qquad 
M_2=S^{2}\;(\text{varianza}).
\]
\end{frame}

\begin{frame}{Momentos para datos agrupados}
Si las marcas de clase son \(y_i\) con frecuencias \(f_i\) y \(n=\sum f_i\):
\[
\hat{y}^{\,i}=\frac{1}{n}\sum_{i} f_i\,y_i^{\,i},
\qquad
M_i=\frac{1}{n}\sum_{i} f_i\,\bigl(y_i-\bar{y}\bigr)^{i}.
\]
\end{frame}

% ==============================
\subsection{Asimetría}
% ==============================
\begin{frame}{Asimetría o sesgo}
La forma de una distribución puede compararse con moda, mediana y media.
\begin{itemize}
\item Si la media \(>\) mediana (y/o moda), \textbf{asimetría positiva} (sesgo a la derecha).
\item Si la media \(<\) mediana (y/o moda), \textbf{asimetría negativa} (sesgo a la izquierda).
\item Si es simétrica, la asimetría es \textbf{nula}.
\end{itemize}
\medskip

\textbf{Coeficiente de asimetría (mediana):}
\[
\text{Sesgo} = \frac{3\,(\bar{x}-\mathrm{Me})}{S}.
\]
\end{frame}

% ==============================
\subsection{Curtosis}
% ==============================
\begin{frame}{Curtosis o apuntalamiento}
Mide cuán \textbf{puntiaguda} o \textbf{aplanada} es la forma de la distribución.
\begin{itemize}
\item \textbf{Leptocúrtica:} más puntiaguda/estrecha que una normal.
\item \textbf{Mesocúrtica:} referencia intermedia (p.\,ej.\ normal).
\item \textbf{Platicúrtica:} más aplanada/anchada.
\end{itemize}
\[
b=\frac{M_{4}}{S^{4}} \;=\; \frac{M_{4}}{M_{2}^{2}},
\qquad 
\begin{cases}
b>0 & \text{leptocúrtica},\\
b=0 & \text{mesocúrtica},\\
b<0 & \text{platicúrtica}.
\end{cases}
\]
\end{frame}

\begin{frame}{Curtosis: coeficiente percentil}
Otra medida basada en cuartiles y percentiles:
\[
K=\frac{Q}{P_{90}-P_{10}},
\qquad 
Q=\tfrac{1}{2}\,(Q_3-Q_1)
\quad\text{(rango semi--intercuartil)}.
\]
\end{frame}

\begin{frame}{Esquema visual de curtosis}
\centering
\begin{tikzpicture}[scale=0.8]
% Ejes guía
\draw[gray!40] (-0.2,0) -- (11,0);

% Funciones tipo (no probabilidades exactas: esquema comparativo)
\def\meso{0.7}
\def\lepto{1.2}
\def\plati{0.35}

% Platicúrtica (ancha)
\draw plot[smooth, domain=0:3, samples=200] (\x, {exp(-(\x-1.5)^2/\plati)}) node at (1.5,-0.25) {Platicúrtica};

% Leptocúrtica (angosta)
\draw plot[smooth, domain=4:7, samples=200] (\x, {exp(-(\x-5.5)^2/\lepto)}) node at (5.5,-0.25) {Leptocúrtica};

% Mesocúrtica (media)
\draw plot[smooth, domain=8:11, samples=200] (\x, {exp(-(\x-9.5)^2/\meso)}) node at (9.5,-0.25) {Mesocúrtica};

% Líneas centrales
\draw[gray!60] (1.5,0) -- (1.5, {exp(-(0)^2/\plati)});
\draw[gray!60] (5.5,0) -- (5.5, {exp(-(0)^2/\lepto)});
\draw[gray!60] (9.5,0) -- (9.5, {exp(-(0)^2/\meso)});
\end{tikzpicture}

\medskip
\small (Curvas esquemáticas sólo para ilustrar \emph{forma}.)
\end{frame}
\end{document}