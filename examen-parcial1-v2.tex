\documentclass[11pt,paper=a4,answers,addpoints]{exam}
\usepackage{graphicx,lastpage,comment}
\usepackage{upgreek}
\usepackage{censor}
\censorruledepth=-.2ex
\censorruleheight=.1ex
\hyphenpenalty 10000
\usepackage[paperheight=10.5in,paperwidth=8.27in,bindingoffset=0in,left=0.8in,right=0.8in, top=1.3in,bottom=1in,headsep=.5\baselineskip]{geometry}
\flushbottom
\usepackage[normalem]{ulem}
\usepackage[utf8]{inputenc}
\usepackage[spanish]{babel}
\renewcommand\ULthickness{2pt}
\setlength\ULdepth{1.5ex}
\renewcommand{\baselinestretch}{1}
\pagestyle{empty}
\usepackage[table]{xcolor}

\pagestyle{headandfoot}
\headrule
\newcommand{\continuedmessage}{%
  \ifcontinuation{\footnotesize Pregunta \ContinuedQuestion\ continua\ldots}{}%
}
\runningheader{\footnotesize Programa de Ingeniería de Sistemas}
{\footnotesize Examen 1 Parcial}
{\footnotesize Página \thepage\ de \numpages}
\footrule
\footer{\footnotesize }
{}
{\ifincomplete{\footnotesize Pregunta \IncompleteQuestion\ continua en la siguiente página \ldots}
  {\iflastpage{\footnotesize Final del examen}{\footnotesize Por favor vea la siguiente página\ldots}}}
\usepackage{amsfonts,amsmath}
\usepackage{cleveref}
\crefname{figure}{figura}{figuras}
\crefname{question}{pregunta}{preguntas}
\renewcommand\thequestion{\arabic{question}}
\renewcommand{\questionlabel}{\thequestion)}
\renewcommand{\questionshook}{%
  \setlength{\leftmargin}{0pt}%
  \setlength{\labelwidth}{-\labelsep}%
}
\decimalpoint
\nopointsinmargin
\pointpoints{punto}{puntos}
\marginpointname{\points}
\pointformat{\boldmath\themarginpoints}
\bracketedpoints
\usepackage{quoting,xparse}
\pointpoints{punto}{puntos}
\bonuspointpoints{punto extra}{puntos extra}
\totalformat{Pregunta \thequestion: \totalpoints puntos}
\hqword{Pregunta}
\hpgword{Página}
\hpword{Puntos}
\hsword{Puntos obtenidos}
\htword{Total}
\usepackage{circuitikz}
\usepackage{color}
\usepackage{pgfplots,graphicx}
\pgfplotsset{compat=1.18}
\usepackage{mathrsfs}
\newcommand{\Laplace}[1]{\ensuremath{\mathscr{L}{\left\lbrace #1\right\rbrace}}}
\newcommand{\InvLap}[1]{\ensuremath{\mathscr{L}^{-1}{\left\lbrace #1\right\rbrace}}}

\usepackage{background}
\backgroundsetup{
  scale=1,
  color=black,
  opacity=1,
  angle=0,
  pages=all,
  position=current page.south west,
  nodeanchor=south west,
  vshift=0mm,
  hshift=0mm,
  contents={\includegraphics[width=\paperwidth,height=\paperheight]{fondo.pdf}}
}

\begin{document}
\noprintanswers
\shorthandoff{<>}
\thispagestyle{empty}

%==================== CABECERA ====================
\begin{center}
    \textit{\textbf{Estadística Aplicada — Examen Parcial (Versión B)}}\\
    \small Distribución de Frecuencia · Medidas de Dispersión · Teorema de Bayes
\end{center}
\vspace{-.1cm}
\noindent
\begin{minipage}[t]{.7\textwidth}%
  {\bfseries Nombres}: \makebox[.75\textwidth]{\hrulefill} \par
  {\bfseries Apellidos}: \makebox[.75\textwidth]{\hrulefill} \par
  {\bfseries Docente}: Ricardo Largaespada
\end{minipage}%
\hfill
\begin{minipage}[t]{.4\textwidth}%
  {\bfseries Curso}: Estadística Aplicada \par
  {\bfseries Fecha}: \makebox[2.5cm]{\hrulefill} \par
  {\bfseries Grupo}: \rule{2.8cm}{.4pt}
\end{minipage}
\par\noindent\rule{\textwidth}{1pt}

\noindent\textbf{Instrucciones.} Justifique cada resultado.\hfill Total: 15 puntos.

%==================== INICIO PREGUNTAS ====================
\begin{questions}

%==================== 1. DISTRIBUCIÓN DE FRECUENCIA ====================
\question[5] \textbf{Distribución de frecuencia (horas de sueño).}

Un grupo de \textbf{20 estudiantes universitarios} registró cuántas horas durmieron en una noche normal de semana. Los datos (en horas) son los siguientes:\[5.7,\ 6.2,\ 7.4,\ 8.0,\ 6.3,\ 7.8,\ 7.0,\ 6.1,\ 5.6,\ 8.2,\ 7.5,\ 6.8,\ 7.3,\ 6.5,\ 7.2,\ 8.4,\ 5.9,\ 6.9,\ 7.1,\ 7.7
\]
\begin{parts}
  \part[2] Construya una \textbf{tabla de distribución de frecuencias} utilizando \textbf{5 clases}.
\begin{center}
\begin{tabular}{|c|p{1.5cm}|p{1.5cm}|p{2.0cm}|p{1.5cm}|p{1.5cm}|p{1.5cm}|p{1.5cm}|p{1.5cm}|}
\hline
\textbf{Clase} & \textbf{Límite inferior} & \textbf{Límite superior} & \textbf{Marca de clase (\(\bar{x}\))} & \textbf{$f$} &  \textbf{$F$} & \textbf{$f\cdot x$} & \textbf{$f_r$} &  \textbf{$F_r$}\\
\hline
1 & & & & & & & & \\
\hline
2 & & & & &  & & &\\
\hline
3 & & & & &  & & &\\
\hline
4 & & & & &  & & &\\
\hline
5 & & & & &  & & &\\
\hline
 \multicolumn{6}{|r|}{\cellcolor{gray!20}\textbf{Total}} & & \cellcolor{gray!20} {\bf Total} &\\
\hline
\end{tabular}
\end{center}

  \part[2] Calcule la \textbf{media, mediana y moda} a partir de la tabla.
  \fillwithdottedlines{4cm}
  \part[1] Determine \textbf{cuántas horas como máximo duermen el 75\% de los estudiantes.}
  \fillwithdottedlines{3cm}
\end{parts}

%==================== 2. MEDIDAS DE DISPERSIÓN ====================
\newpage
\question[5]\textbf{Comparación entre grupos (medidas de dispersión).}

Se registró el tiempo (en horas) que dos grupos de \textbf{programadores} dedicaron a depurar código durante una semana:
\[
\text{Grupo A: } 4.1,\ 4.5,\ 4.9,\ 5.1,\ 5.3,\ 5.5,\ 5.7,\ 6.0
\]
\[
\text{Grupo B: } 3.9,\ 4.2,\ 4.5,\ 4.8,\ 5.1,\ 5.5,\ 5.8,\ 6.1
\]

\begin{parts}
  \part[2\half] Para cada grupo, calcule la \textbf{media}, \textbf{varianza}, \textbf{desviación estándar}.
  \fillwithdottedlines{3cm}
  \part[2\half] Compare los resultados y determine \textbf{qué grupo muestra un desempeño más consistente} en sus tiempos de depuración. Considere el \textbf{coeficiente de variación}.
  \fillwithdottedlines{2cm}
\end{parts}

%==================== 3. BAYES ====================
\question[5] \textbf{Teorema de Bayes (detección de spam).}

Un sistema de correo intenta detectar mensajes de \textbf{spam}. Del total de correos recibidos:
\[
P(S)=0.06 \quad (\text{correos realmente spam}), \qquad P(S')=0.94 \quad (\text{correos legítimos})
\]
El filtro de spam tiene las siguientes características:
\[
P(\text{Marcado}\mid S)=0.95, \qquad P(\text{Marcado}\mid S')=0.015.
\]

\begin{parts}
  \part[2\half] Calcule \(P(\text{Marcado})\) aplicando la \textbf{ley de la probabilidad total}.
  \fillwithdottedlines{3cm}
  \part[2\half] Determine la probabilidad de que un correo \textbf{marcado como spam} sea realmente spam, es decir \(P(S\mid \text{Marcado})\).
  \fillwithdottedlines{2cm}
\end{parts}

\end{questions}
\end{document}
